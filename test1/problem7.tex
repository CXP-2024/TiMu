\section{Euler Method (35pts)}
弹簧-质量模型是一个简单的物理系统,由连接到弹簧上的质量组成。质量可以来回移动,这是由弹簧根据胡克定律施加的恢复力引起的。这是一个可以通过数值方法模拟的经典力学系统。\\
质量为$m$的物体连接到弹性常数为$k$的弹簧上的基本运动方程为:
\[
m \frac{d^2x}{dt^2} + kx = 0
\]
这个方程是一个二阶微分方程,我们可以使用数值方法来近似求解这个方程。为此,我们需要离散化时间并用有限差分代替连续导数。我们将考虑求解这类微分方程的两种常见方法:\textbf{显式欧拉法}和\textbf{隐式欧拉法}。
\subsection*{显式欧拉法}
显式欧拉法是近似求解的最简单方法之一。为了应用它,我们需要离散化速度和位置。设$v(t)$为速度,因此:
\[
v(t) = \frac{dx}{dt}, \quad a(t) = \frac{dv}{dt} = \frac{d^2x}{dt^2}
\]
离散时间步骤表示为$t_n = n \Delta t$,其中$\Delta t$是时间步长。使用显式欧拉法:
\begin{itemize}
	\item 位置更新:$x_{n+1} = x_n + v_n \Delta t$;    速度更新:$v_{n+1} = v_n + a_n \Delta t$
\end{itemize}
\subsection*{隐式欧拉法}
隐式欧拉法比显式方法更稳定,尤其是对于刚性系统。它使用下一时间步的值,而不是当前时间步的值来计算更新。隐式欧拉法更新位置和速度如下:
\begin{itemize}
	\item 位置更新:$x_{n+1} = x_n + v_{n+1} \Delta t$;    速度更新:$v_{n+1} = v_n + a_{n+1} \Delta t$
\end{itemize}
设$s_n = [x_n, v_n]$,$s_0 = [x_0, v_0]$为初始状态。
\begin{enumerate}
	\item 基于胡克定律使用$x_n$定义$a_n$。(5pts)
	\item 推导使用显式欧拉法的$\{s_n\}$的通用公式.(20pts)
	\item 推导使用隐式欧拉法的$\{s_n\}$的通用公式.(10pts)
	\item 思考:解释为什么"隐式欧拉法比显式方法更稳定"。隐式欧拉法可能有哪些缺点?(bonus: 0pts)
\end{enumerate}
