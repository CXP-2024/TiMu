\section{地脉热管(50pts)}
近日,Teyvat地脉深处发生了一系列紊乱,导致地脉热管的异常升温。这些热管是连接地脉与地表的重要通道,它们的温度升高可能会引发一系列地质灾害。地脉流动着许多的熔岩物质,这些熔岩的密度为\(\rho(\vec{r,t})\),比热容为\(c\),流速为\(\vec{v}(r,t)\),流动的熔岩在地脉中具有许许多多的热量交换。已知傅里叶定律为:\(\vec{j}_c = -k \nabla T\)。为单位时间单位面积内传导的热量,\(k\)为导热系数,为常数。下面你均只需要考虑热传导和热对流,不考虑其他因素的影响。
\begin{enumerate}
	\item 已知地脉中的温度分布为\(T(\vec{r,t})\),请你推导出地脉中熔岩的热量流动方程(5pts)
	\item 现在考虑连接地脉与地表的热管,为简单起见,我们将热管视为一个圆柱体,半径为\(R\),长度为\(L\),底面与地脉的接触面积为\(S\),热管底部与顶部的温度分别为\(T_1\), \(T_2\),假设热管熔岩中的温度分布已经达到了稳态,并假设热管中的熔岩流速为常数向上的\(v\),并设热管具有散热功能,单位时间单位面积散热量为\(\alpha T\),请你推导出热管中熔岩的温度分布\(T(x)\),\(x\)为沿着热管的轴向坐标,从下至上为\(0\)到\(L\)。(30pts)
	\item 并求出该热管的总散热量\(Q_c\)。(10pts)
	\item 现在考虑一个二维的流动着的稳态柱对称的熔岩,其流速径向向外流动\(v(r)\),设原点有源源不断的熔岩流出,流量为\(Q\),请你推导出熔岩的温度分布\(T(r)\)满足的微分方程。(5pts)
\end{enumerate}