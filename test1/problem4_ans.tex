\section{地脉热管(50pts)}
近日,Teyvat地脉深处发生了一系列紊乱,导致地脉热管的异常升温。这些热管是连接地脉与地表的重要通道,它们的温度升高可能会引发一系列地质灾害。地脉流动着许多的熔岩物质,这些熔岩的密度为\(\rho(\vec{r,t})\),比热容为\(c\),流速为\(\vec{v}(r,t)\),流动的熔岩在地脉中具有许许多多的热量交换。已知傅里叶定律为:\(\vec{j}_c = -k \nabla T\)。为单位时间单位面积内传导的热量,\(k\)为导热系数,为常数。下面你均只需要考虑热传导和热对流,不考虑其他因素的影响。
\begin{enumerate}
	\item 已知地脉中的温度分布为\(T(\vec{r,t})\),请你推导出地脉中熔岩的热量流动方程(5pts)
	\item 现在考虑连接地脉与地表的热管,为简单起见,我们将热管视为一个圆柱体,半径为\(R\),长度为\(L\),底面与地脉的接触面积为\(S\),热管底部与顶部的温度分别为\(T_1\), \(T_2\),假设热管熔岩中的温度分布已经达到了稳态,并假设热管中的熔岩流速为常数向上的\(v\),并设热管具有散热功能,单位时间单位面积散热量为\(\alpha T\),请你推导出热管中熔岩的温度分布\(T(x)\),\(x\)为沿着热管的轴向坐标,从下至上为\(0\)到\(L\)。(30pts)
	\item 并求出该热管的总散热量\(Q_c\)。(10pts)
	\item 现在考虑一个二维的流动着的稳态柱对称的熔岩,其流速径向向外流动\(v(r)\),设原点有源源不断的熔岩流出,流量为\(Q\),请你推导出熔岩的温度分布\(T(r)\)满足的微分方程。(5pts)
\end{enumerate}

\section*{Answer 4}
\begin{enumerate}
	\item 空间中的热流为:
	\begin{align*}
		\vec{j} = \vec{j}_c + \vec{j}_d = -k \nabla T + \rho c T\vec{v} 
	\end{align*}
	做一个高斯面积分,得到:
	\begin{align*}
		\frac{\partial}{\partial t} \int_V \rho c T dV + \int_S \vec{j} \cdot dS = 0
	\end{align*}
		根据高斯定理,有:
	\begin{align*}
		\int_S \vec{j} \cdot dS = \int_V \nabla \cdot \vec{j} dV
	\end{align*}
		因此有:
	\begin{align*}
		\frac{\partial (\rho c T)}{\partial t} = k \nabla^2 T -  \nabla \cdot(\rho c T\vec{v})
	\end{align*}
		
	\item 设热管中熔岩的温度分布为\(T(x)\),则根据稳态条件,有:
	\begin{align*}
		j(x) &= -k \frac{\partial T}{\partial x} + \rho c v T(x) \\
		j'(x)S &= -2\pi R \alpha T(x) 
	\end{align*}
	整理得到微分方程为:
	\begin{align*}
		-kS T''(x) + \rho cS v T'(x) = -2\pi R \alpha T(x)
	\end{align*}
	两个特征根为:
	\begin{align*}
		\lambda_{1,2} &= \frac{\rho c v}{2k} \pm \sqrt{(\frac{\rho c v}{2k})^2 + \frac{2\pi R \alpha}{kS}}
	\end{align*}
		因此通解为:
	\begin{align*}
		T(x) &= A e^{\lambda_1 x} + B e^{\lambda_2 x} \\
		j(x) &= -k \frac{\partial T}{\partial x} + \rho c v T(x)
	\end{align*}
	带入边界条件:
	\begin{align*}
		T(0) &= T_1 \\
		T(L) &= T_2
	\end{align*}
		得到:
	\begin{align*}
		A &= \frac{T_2 - T_1 e^{\lambda_2 L}}{e^{\lambda_1 L} - e^{\lambda_2 L}} \\
		B &= \frac{T_1 e^{\lambda_1 L} - T_2 }{e^{\lambda_1 L} - e^{\lambda_2 L}}
	\end{align*}
		热管的总散热量为:
	\begin{align*}
		Q_c &= j(0)S-j(L)S \\
		&= [-kS \frac{\partial T(x)}{\partial x} + \rho c Sv T(x)]_{0}^L \\
		&= AS(e^{\lambda_1 L} - 1)(-k\lambda_1+c\rho v) + BS(e^{\lambda_2 L} - 1)(-k\lambda_2+c\rho v) 
	\end{align*}
	\item 先求流速的表达式:
	\begin{align*}
		Q &= 2\pi R v(r) \\
		v(r) &= \frac{Q}{2\pi r}
	\end{align*}
	带入(1)的方程,得到:
	\begin{align*}
		 k T''(r) -  \frac{Q\rho c}{2\pi } \frac{d}{d r} (\frac{T(r)}{r})=0\\
	\end{align*}



\end{enumerate}